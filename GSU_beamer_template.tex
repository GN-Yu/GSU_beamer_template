\documentclass{beamer}
\usepackage{xcolor}
\usepackage{amsmath,amsfonts}
\usepackage{appendixnumberbeamer}

\usetheme{Warsaw} % Alternatively: Copenhagen, Warsaw

\definecolor{GSUThemeBlue}{rgb}{0.0,0.22,0.65}
\definecolor{GSURedAccent}{rgb}{0.78,0.05,0.19}
\definecolor{GSUBlueSteel}{rgb}{0.22,0.25,0.34}
\definecolor{GSUCoolBlue}{rgb}{0.0,0.44,0.81}
\definecolor{GSUVibrantBlue}{rgb}{0.0,0.69,0.94}
\definecolor{GSULightBlue}{rgb}{0.59,0.79,0.92}
\definecolor{GSULightGray}{rgb}{0.93,0.93,0.93}
\definecolor{GSUMediumGray}{rgb}{0.8,0.8,0.8}
\definecolor{GSUDarkGray}{rgb}{0.47,0.46,0.47}

%\usecolortheme[named=GSUThemeBlue]{structure}
\setbeamercolor{palette primary}{bg=GSUThemeBlue,fg=white}
\setbeamercolor{palette secondary}{bg=GSUThemeBlue,fg=white}
\setbeamercolor{palette tertiary}{bg=GSUBlueSteel,fg=white}
\setbeamercolor{palette quaternary}{bg=GSULightBlue,fg=white}
\setbeamercolor{structure}{fg=GSUThemeBlue} % itemize, enumerate, etc
\setbeamercolor{section in toc}{fg=GSUBlueSteel} % TOC sections
%
%% Override palette coloring with secondary
\setbeamercolor{subsection in head/foot}{bg=GSUThemeBlue,fg=white}
\setbeamercolor{block title alerted}{bg=GSURedAccent,fg=white}
\setbeamercolor{block title}{bg=GSUThemeBlue,fg=white}
\setbeamercolor{alerted text}{fg=GSURedAccent}

\useoutertheme{infolines} % Alternatively: default, miniframes, infolines, split, shadow
\useinnertheme{circles}

\usefonttheme[onlymath]{serif}
\usefonttheme[onlylarge]{structurebold} % Alternatively in {}: structurebold, structureitalicserif, structuresmallcapsserif; in []: onlylarge, onlysmall

%%% footline with no page number
%\setbeamertemplate{footline}
%{%
%	\leavevmode
%	\begin{beamercolorbox}[wd=.5\paperwidth,ht=2.25ex,dp=1ex,center]{author in head/foot}%
%		\insertshortauthor{}
%	\end{beamercolorbox}%
%	\begin{beamercolorbox}[wd=.5\paperwidth,ht=2.25ex,dp=1ex,center]{subsection in head/foot}%
%		\insertshorttitle
%	\end{beamercolorbox}%
%}


% footline with page number (spacing may need some maual adjustments)
\setbeamertemplate{footline}
{%
	\leavevmode
	\begin{beamercolorbox}[wd=.5\paperwidth,ht=2.25ex,dp=1ex,center]{author in head/foot}%
		\insertshortauthor{}
	\end{beamercolorbox}%
	\begin{beamercolorbox}[wd=.5\paperwidth,ht=2.25ex,dp=1ex,left]{subsection in head/foot}%
		\hspace*{.2\paperwidth}\insertshorttitle\hspace*{.15\paperwidth}\insertframenumber/\inserttotalframenumber	% adjust page number spacing here
	\end{beamercolorbox}%
}


\title[A Short title]{Your Presentation Title}
\date{\today}
\author[F. L.]{Firstname Lastname}
\institute[GSU]{Georgia State University}
\titlegraphic{\includegraphics[width=2.5cm]{fig/PrimaryLogo3color.jpg}}
\begin{document}
	\begin{frame}
		\titlepage
	\end{frame}
	\begin{frame}
		\tableofcontents
	\end{frame}

	\section{First Section}
	\begin{frame}{First Frame}
		Hello, world!
	\end{frame}
	\begin{frame}{Color Table}
		
		\begin{columns}
			\begin{column}{0.33\textwidth}
				\begin{itemize}
					\item \textcolor{GSUThemeBlue}{GSU Theme Blue}
					\item \textcolor{GSUCoolBlue}{GSU Cool Blue}
					\item \textcolor{GSUVibrantBlue}{GSU Vibrant Blue}
					\item \textcolor{GSULightBlue}{GSU Light Blue}
				\end{itemize}
			\end{column}
			
			\begin{column}{0.33\textwidth}
				\begin{itemize}
					\item \textcolor{GSUBlueSteel}{GSU Blue Steel}
					\item \textcolor{GSUDarkGray}{GSU Dark Gray}
					\item \textcolor{GSUMediumGray}{GSU Medium Gray}
					\item \textcolor{GSULightGray}{GSU Light Gray}
				\end{itemize}
			\end{column}
		
			\begin{column}{0.33\textwidth}
				\begin{itemize}
					\item \textcolor{GSURedAccent}{GSU Red Accent}
				\end{itemize}
			\end{column}
		\end{columns}
		
	\end{frame}
	
	\begin{frame}{Font size in Beamer (default)}
		\tiny This is tiny font size
		\scriptsize This is scriptsize font size
		\footnotesize This is footnotesize font size
		\small This is small font size
		\normalsize This is normalsize font size
		\large This is large font size
		\Large This is Large font size
		\LARGE This is LARGE font size
		\huge This is huge font size
		\Huge This is Huge font size
	\end{frame}

	\section{Second Section}
	\begin{frame}[fragile]{Blocks}
		\begin{block}{Theorem 1}
			This is a theorem.
		\end{block}
		\begin{proof}
			Here you can briefly prove your Theorem 1.
			
			A proof is better not to be very long. Use \verb|\alert| to highlight \alert{important things}.
		\end{proof}
		
		\begin{exampleblock}{Example 1}
			This is an example.
		\end{exampleblock}
		
		\begin{alertblock}{Alert 1}
			Add \verb|[fragile]| after \verb|\begin{frame}| when using verbatim environments.
		\end{alertblock}
	\end{frame}

	\section{Third Section}
	\begin{frame}{Serif fonts in math}
		A Dirichlet distribution is a distribution over the $K$-dimensional probability simplex:
		$$
		\Delta_K=\left\{\left(\pi_1, \ldots, \pi_K\right): \pi_k \geq 0, \sum_k \pi_k=1\right\}
		$$
		We say $\left(\pi_1, \ldots, \pi_K\right)$ is Dirichlet distributed,
		$$
		\left(\pi_1, \ldots, \pi_K\right) \sim \operatorname{Dirichlet}\left(\alpha_1, \ldots, \alpha_K\right)
		$$
		with parameters $\left(\alpha_1, \ldots, \alpha_K\right)$, if
		$$
		p\left(\pi_1, \ldots, \pi_K\right)=\frac{\Gamma\left(\sum_k \alpha_k\right)}{\prod_k \Gamma\left(\alpha_k\right)} \prod_{k=1}^K \pi_k^{\alpha_k-1}
		$$
	\end{frame}

	\section{Fourth Section}
	\begin{frame}{Notes if two screen allows}
		\begin{itemize}
			\item<1-> Eggs
			\item<2-> Plants
			\note[item]<2>{Tell joke about plants.}
			\note[item]<2>{Make it short.}
			\item<3-> Animals
		\end{itemize}
	\end{frame}

	\begin{frame}{Highlighting the Current Item in an Enumeration}
		\begin{itemize}
			\item<1-| alert@1> First point.
			\item<2-| alert@2> Second point.
			\item<3-| alert@3> Third point.
		\end{itemize}

	\end{frame}

	\begin{frame}{Uncovering Tagged Formulas Piecewise}
		\begin{align*}
			A &= B \\
			\uncover<2->{&= C \\}
			\uncover<3->{&= D \\}
		\end{align*}
	\end{frame}

	\begin{frame}[fragile]{An Algorithm For Finding Primes Numbers.}
		\begin{semiverbatim}
			\uncover<1->{\alert<1>{int main (void)}}
			\uncover<1->{\alert<1>{\{}}
			\uncover<2->{\alert<2>{	\alert<5>{std::}vector<bool> is_prime (100, true);}}
			\uncover<2->{\alert<2>{ for (int i = 2; i < 100; i++)}}
			\uncover<3->{\alert<3>{ 	if (is_prime[i])}}
			\uncover<3->{\alert<3>{ 		\{}}
			\uncover<4->{\alert<4>{ 		\alert<5>{std::}cout << i << " ";}}
			\uncover<4->{\alert<4>{ 		for (int j = i; j < 100;}}
			\uncover<4->{\alert<4>{ 				is_prime [j] = false, j+=i);}}
			\uncover<3->{\alert<3>{ 		\}}}
			\uncover<1->{\alert<1>{ return 0;}}
			\uncover<1->{\alert<1>{\}}}
		\end{semiverbatim}
		\visible<5->{Note the use of \alert{\texttt{std::}}.}
	\end{frame}
	
	\section*{} 
	\begin{frame}{}
		\centering \Huge
		\emph{Thanks!}
	\end{frame}

	\appendix
	\begin{frame}{Backup Slide1}
		You can add some backup slides for expected questions from the audience.
	\end{frame}

	\begin{frame}{Backup Slide2}
		These backup slides are independently numbered.
	\end{frame}
	
\end{document}